% This file is part of the stream_information project.
% Copyright 2017 the authors. All rights reserved.

% # style notes
% - it is Cram\'er--Rao not Cram\'er-Rao. And yet Fisher-matrix not Fisher--matrix.

% TODO:
% - Reference for the dustmaps package? See software section.

\documentclass[modern]{aastex62}

\usepackage{amsmath}

% typography
\setlength{\parindent}{1.\baselineskip}
% \newcommand{\acronym}[1]{{\small{#1}}}
\newcommand{\package}[1]{\textsl{#1}}
\newcommand{\gaia}{\textsl{Gaia}}

% aastex parameters
% \received{not yet; THIS IS A DRAFT}
%\revised{not yet}
%\accepted{not yet}
% % Adds "Submitted to " the arguement.
% \submitjournal{ApJ}
\shorttitle{GD-1 in Gaia DR2}
\shortauthors{price-whelan \& bonaca}

%@arxiver{}

\begin{document}\sloppy\sloppypar\raggedbottom\frenchspacing % trust me

\title{Gaia DR2 view of the GD-1 stellar stream}

\author[0000-0003-0872-7098]{Adrian~M.~Price-Whelan}
\affiliation{Department of Astrophysical Sciences,
             Princeton University, Princeton, NJ 08544, USA}
\email{adrn@astro.princeton.edu}
\correspondingauthor{Adrian M. Price-Whelan}

\author[0000-0002-7846-9787]{Ana Bonaca}
\affil{Harvard--Smithsonian Center for Astrophysics, Cambridge, MA 02138, USA}


\begin{abstract}\noindent % trust me

\end{abstract}

\keywords{Galaxy: halo --- dark matter}

\section{Introduction}
\label{sec:intro}


\section{Data}
\label{sec:data}


Most of the clearly identifiable portions of the GD-1 stream are located at high
Galactic latitudes ($b > 35^\circ$), and we therefore do not expect significant
dust extinction or variations in extinction along the stream.
\figurename~\ref{fig:sfd} shows the $V$-band extinction in the region around the
GD-1 stream, computed from the Schlegel-Finkbeiner-Davis extinction map
(\cite{Schlegel:1998}; hereafter SFD).
Vertical lines indicate the two regions clearly identifiable as under-densities
from the proper-motion-selected stream members.
The maximum $A_V$ is $\approx$0.07 mag in both the under-density and gap
regions.
The dispersion in $A_V$ in the region $-60^\circ < \phi_1 < 0^\circ$ and
$-1^\circ < \phi_1 < 1^\circ$ is $\std_{A_V} \approx 0.03~\textrm{mag}$.

% Notebook: GD1-dust-and-completeness
\begin{figure}[h]
\begin{center}
\includegraphics[width=\textwidth]{sfd.pdf}
\end{center}
\caption{%
Colored background shows the \gaia\ $G$-band extinction, $A_G$, in the GD-1
coordinate system.
Vertical lines roughly show the regions identified as the under-density (UD) and
gap.
The maximum extinction in the UD or Gap regions is $\approx$0.07 mag.
\label{fig:sfd}
}
\end{figure}


\section{Results}
\label{sec:results}

\subsection{Global properties}
\label{sec:res_global}

\subsection{Gap}
\label{sec:res_gap}

\subsection{Underdensity}
\label{sec:res_underdensity}


\section{Discussion}
\label{sec:discussion}


\acknowledgements{
thanks: Belokurov, Casey, Geha, Hogg, Johnston, Koposov, Schlafly, Spergel
This research was started at the NYC Gaia DR2 Workshop at the Center for Computational Astrophysics of the Flatiron Institute in 2018 April.
AB acknowledges generous support from the Institute for Theory and Computation at Harvard University.
All code used in this work and all results are available at \url{https://github.com/adrn/GD1-DR2}.
}

\software{
    \package{Astropy} \citep{astropy},
    \package{dustmaps}\footnote{\url{https://github.com/gregreen/dustmaps}},
    \package{gala} \citep{gala},
    \package{IPython} \citep{ipython},
    \package{matplotlib} \citep{mpl},
    \package{numpy} \citep{numpy},
    \package{scipy} \citep{scipy}
}

\bibliographystyle{aasjournal}
\bibliography{gd1}

\clearpage

\appendix
\section{Completeness check and data validation}
\label{sec:validate}

% % Notebook:
% \begin{figure}[h]
% \begin{center}
% \includegraphics[width=0.7\textwidth]{nvisits.pdf}
% \end{center}
% \caption{%
% TODO
% \label{fig:TODO}
% }
% \end{figure}


\end{document}
